\section{Introduction to JADE experiment}
The description of JADE experiment can be found elsewhwre.\cite{Bartel:1986ua}
Physics cases for the available JADE data can be discussed elsewhere\cite{Bartel:1986ua}.
The case of JADE data preservation can be found interesting as a mode of data preservation that can be used for other experimnts.
The main part of the preservation is the  software, data and documentation.
\section{JADE software}
JADE software description can be found in the JADE computing notes and elsewhwre.
Briefly it consists of approximately 50.000 lines of IBM Fortran77 and similar codes 
that were compiled on AIX4.3 systems with IBM Fortran compiler and corresponding runtime.
The dependancies include CERNLIB.


\subsection{Porting JADE software}
This section describes the process of porting JADE software to modern systems.
Please skip it if you are not interested.
As of 2016 these systems are not available on the market\footnote{aapart from few on e-Bay},
therefor e the idea is to port the software on modern wide 
The main problems with the porting software to the modern systems are
\begin{itemize}
\item  compiler-specific Fortran features
\item  big-little endianess for the calculs
\item  big-little endianess for the I/O
\item  build system 
\item  porting CERNLIB
\end{itemize}

The following platforms were considered:
\setlength{\tabcolsep}{0.0em}
%\begin{table}[htbp]\tiny\centering
\begin{sidewaystable}[htbp]\scriptsize\centering
\begin{tabular}{|c|c|c|c|c|c|c|c|c|c|}\hline
    & Origin/Now        & Hardware, num               &        &OS             & arch(bits)               & Fortran/FC                &Build system              &GUI      & CERNLIB/runtime  \\\hline\hline
A   & 198x/Dead         &  IBM/370,Nord?              & ?      &?              &   ?(32be)                & 77?/xlf?                  &?                         &NORD? & DESYLIB/?     \\
B   & 2003/Not tried    &  RS6000, 1                  & ?      &AIX4.3         & ppc(32be)                & 77/xlf 7.1                &make 3.78 and ksh scripts &HIGZ  & 2002/?/32bit     \\
C   &      Not tried    &  RS6000, 1                  & ?      &AIX4.3         & ppc(32be)                & 77/g77 0.5 (gcc2.95)      &make 3.78 and ksh scripts &HIGZ      & 2002/?/32bit     \\
D   &      Not avail.   &  RS6000, 1                  & ?      &AIX4.3         & ppc(32be)                & 77/xlf 7.1                &make 3.78 and cmake?      &HIGZ      & 2002/?/32bit     \\
E   &      Not avail.   &  RS6000, 1                  & ?      &AIX4.3         & ppc(32be)                & 77/g77 0.5 (gcc2.95)      &make 3.78 and cmake?      &HIGZ      & 2002/?/32bit     \\
F   & 2016/Runs         &  pSeries\@QEMU2.7, many   & 3/10   &CentOS7.2 & ppc64+ppc(64be+32be)     & 90/xlf 15.1 (trial)       &make 3.81 and cmake 2.8.11       &HIGZ      & 2006/gfortran/32bit     \\
G   & 2016/Runs         &  pSeries\@QEMU2.7, many   & 3/10   &CentOS7.2 & ppc64+ppc(64be+32be)     & 90/gfortran 4.8.5         &make 3.81 and cmake 2.8.11       &HIGZ      & 2006/gfortran/32bit     \\
H   & 2016/Runs         &  Intel/AMD, many            &10/10   &CentOS7.2 & x8664+i686(64le+32le)   & 90/gfortran 4.8.5         &make 3.81 and sh scripts        &HIGZ      & 2006/gfortran/32bit     \\
I   & 2016/Runs         &  Intel/AMD, many            &10/10   &CentOS7.2 & x8664+i686(64le+32le)   & 90/gfortran 4.8.5         &make 3.81 and cmake 2.8.11      &HIGZ      & 2006/gfortran/32bit     \\
J   & 2016/Runs         &  Intel/AMD, many            &10/10   &CentOS7.2 & x8664+i686(64le+32le)   & 90/gfortran 4.8.5         &make 3.81 and cmake 2.8.11      &ROOT/32bit& picocernlib/gfortran/32bit     \\
K   & 2016/Runs         &  Intel/AMD, many            &10/10   &CentOS7.2 & x8664(64le)             & 90/gfortran 4.8.5         &make 3.81 and cmake 2.8.11      &ROOT/64bit& picocernlib/gfortran/64bit\\
\hline
\end{tabular}
\caption{Status of different platforms.  Picocernlib is a small set of CERNLIB routines plus some ROOT backend fro plotting. }
\label{tab:status}
\end{sidewaystable}
%\end{table}


Because of cernlib issues everything was tested in 32 bit mode.
Maybe that would worth to copy the needed cernlib routines to have them separately?
Will minicernlib work?The hardest part is graphics.

The MC generation can be done on any platform and put into HepMC3 format.
Practicaly it is visible for G and H.
The conversion to any be and le JADE MC formats can be done on any platform between F and J,
as HepMC3 works only with gcc4 and compatible compilers. 



\subsubsection{CERNLIB}
Compilation of cernlib for CentOS7 and especially in 32bit mode is an issue.
CERNLIB is needed for i686 and ppc architectures.
Both were build using CentOS7.2 64 machines (virtual or real).



The missing packages for i686, cernlib and xbae, were recompiled from CentOS6 source rpms via mock on x8664
\begin{verbatim}
mock
\end{verbatim}



For ppc the mock environment is not available, so the build was done manualy.
All needed 32 bit packages were installed from CentOS7 AltArch repository.
The two missing packages, cernlib and xbae were recompiled from CentOS6 source rpms.
\begin{Verbatim}[fontsize=\tiny]
LDFLAGS=' -L/usr/lib   -Wl,-z -Wl,relro' CC='gcc -m32 ' CXX='g++ -m32 '   rpmbuild --target=ppc  --rebuild xbae-4.60.4-12.el6.src.rpm -D'_lib lib' -D'libdir /usr/lib'
\end{Verbatim}
The cernlib source rpm was patched, but still has some issues, with using user flags, so
a brute force trick is to replace /usr/bin/gfortran with a wrapper with explicit -m32 -L/usr/lib flags.
Same trick for /usr/bin/rm:   add -f option.

The full list of used 32 bit rpms follows (geven for ppc architecture).


\begin{minipage}{0.5\textwidth} 
\begin{Verbatim}[fontsize=\tiny]
blas-3.4.2-5.el7.ppc
blas-devel-3.4.2-5.el7.ppc
cernlib-2006-36.el7.centos.ppc
cernlib-debuginfo-2006-36.el7.centos.ppc
cernlib-devel-2006-36.el7.centos.ppc
cernlib-packlib-gfortran-2006-36.el7.centos.ppc
cernlib-static-2006-36.el7.centos.ppc
cernlib-utils-2006-36.el7.centos.ppc
compat-libstdc++-33-3.2.3-72.el7.ppc
expat-2.1.0-8.el7.ppc
fontconfig-2.10.95-7.el7.ppc
freetype-2.4.11-11.el7.ppc
geant321-2006-36.el7.centos.ppc
glib2-2.42.2-5.el7.ppc
glibc-2.17-106.el7_2.8.ppc
glibc-devel-2.17-106.el7_2.8.ppc
kuipc-2006-36.el7.centos.ppc
lapack-3.4.2-5.el7.ppc
lapack-devel-3.4.2-5.el7.ppc
libffi-3.0.13-16.el7.ppc
libgcc-4.8.5-4.el7.ppc
libgfortran-4.8.5-4.el7.ppc
libgfortran-static-4.8.5-4.el7.ppc
libICE-1.0.9-2.el7.ppc
libjpeg-turbo-1.2.90-5.el7.ppc
libjpeg-turbo-devel-1.2.90-5.el7.ppc
libpng-1.5.13-7.el7_2.ppc
libpng-devel-1.5.13-7.el7_2.ppc
libselinux-2.2.2-6.el7.ppc
libSM-1.2.2-2.el7.ppc
libstdc++-4.8.5-4.el7.ppc
libstdc++-devel-4.8.5-4.el7.ppc
libuuid-2.23.2-26.el7_2.3.ppc
\end{Verbatim}
\end{minipage}
\begin{minipage}{0.5\textwidth} 
\begin{Verbatim}[fontsize=\tiny]
libX11-1.6.3-2.el7.ppc
libX11-devel-1.6.3-2.el7.ppc
libXau-1.0.8-2.1.el7.ppc
libXau-devel-1.0.8-2.1.el7.ppc
libXaw-1.0.12-5.el7.ppc
libXaw-devel-1.0.12-5.el7.ppc
libxcb-1.11-4.el7.ppc
libXext-1.3.3-3.el7.ppc
libXext-devel-1.3.3-3.el7.ppc
libXft-2.3.2-2.el7.ppc
libXft-devel-2.3.2-2.el7.ppc
libXmu-1.1.2-2.el7.ppc
libXmu-devel-1.1.2-2.el7.ppc
libXp-1.0.2-2.1.el7.ppc
libXp-devel-1.0.2-2.1.el7.ppc
libXpm-3.5.11-3.el7.ppc
libXpm-devel-3.5.11-3.el7.ppc
libXrender-0.9.8-2.1.el7.ppc
libXt-1.1.4-6.1.el7.ppc
libXt-devel-1.1.4-6.1.el7.ppc
motif-2.3.4-7.el7.ppc
motif-devel-2.3.4-7.el7.ppc
nss-softokn-freebl-3.16.2.3-14.2.el7_2.ppc
nss-softokn-freebl-devel-3.16.2.3-14.2.el7_2.ppc
patchy-gfortran-2006-36.el7.centos.ppc
paw-gfortran-2006-36.el7.centos.ppc
pcre-8.32-15.el7_2.1.ppc
pkgconfig-0.27.1-4.el7.ppc
xbae-4.60.4-12.el7.centos.ppc
xbae-devel-4.60.4-12.el7.centos.ppc
xz-libs-5.1.2-12alpha.el7.ppc
zlib-1.2.7-15.el7.ppc
zlib-devel-1.2.7-15.el7.ppc
\end{Verbatim}
\end{minipage}

A similar list is needed for the i686 architecture.



\subsubsection{Endianness}
The I/O endianess  can be controled in gfortran runtime with the 
GFORTRAN\_CONVERT\_UNIT environment variable, e.g.
\begin{verbatim}
export GFORTRAN_CONVERT_UNIT='big_endian;native:2'
\end{verbatim}

\subsubsection{cmake}
The cmake file is selfdocumented.
Since version 2 cmake  has a first class Fortran support with descent dependency tracking.
This replaces self developed scripts in kors shell.

\subsubsection{Compiler specific issues}
Few changes were needed after the porting in   was done.
In many places the types were changed from LOGICAL to CHARCTER
and from INTEGER to CHARECTER.

As the line lenghth is not limited, some symbols were removed in the end of lines.

HOLLERITH constants are represented in IBM Fortran via
\begin{verbatim}
'ABCD'
\end{verbatim}
were replaced with
\begin{verbatim}
4HABCD
\end{verbatim}
for GNU fortran. This format is also accepted by the IBM compillers.
Similar issue was with the hexademical constants, which are in the form 
\begin{verbatim}
Z1000
\end{verbatim}
in IBM fortran. These were replaced with 
\begin{verbatim}
Z'1000'
\end{verbatim}
The form is accepted by GNU Fortran and IBM compiuller.
One compiller specific routine was added: hfix (gfortran is missing it.)

\subsubsection{Sources treatment}
The modified  sources were put into a separate directory src2016 and main2016.
Some used scripts are in the directory scripts2016.


\subsection{Using JADE software}
The basic usage of JADE software is described elsewhere.
The documentations stays valid.
Below we describe the environment for JADE and the process of generation of new MC with the JADE software.

\subsubsection{Environment for JADE software}
JADE software does not need special environment to run.
However, a virtual machine with all software installed can be downloaded 
from wwwjade.mpp.mpg.de as well as the installation ISOS.

The basic dependencies of the software are 
\begin{itemize}
\item cmake for building, should be installed
\item C/C++ compiller with basic c++11 support, should be installed
\item fortran compiler with fortran90 or fortran95 support, should be installed
\item ROOT5 or ROOT6 for graphics, histogramming
\item HepMC3 for Monte Carlo, provided in the tarball 
\item lapack for Monte Carlo , should be installed
\end{itemize}
Despite the VM has MC generators installed, the  Monte Carlo HepMC files can be generated elsewhere as well.


\subsubsection{MC generation}

The setup of MC generators is ommited.
The only requirement for the MC enerator is to produce the 
output in HepMC2/HepMC3/HEPEVT format.
After that the utility convert\_example\_JADE.exe can be used.
It can produce JADE MC ASCII  files, as well as JADE MC le and be binary files.

\begin{Verbatim}[fontsize=\tiny]
convert_example_JADE.exe hepmc2_jade  sherpa34gev.hepmc2   bsherpa34gev.jade Mode=0
convert_example_JADE.exe hepmc2_jade  sherpa34gev.hepmc2   asherpa34gev.jade Mode=1
\end{Verbatim}
In the later the sherpa34gev.hepmc2 is the input file in hepmc2 format and the bsherpa34gev.jade is an output in JADE binary format.
The options are given in the help of convert\_example\_JADE.exe.

The resulted files can be used for the superv simulation and reconstruction utility.
\subsubsection{Analysis}
The final  data  is stored in ROOT or HBOOK fromat  and can be analyzed on arbitrary systems with ROOT or PAW.



\section{JADE data}
The JADE data is stored in MPCDF.
It can be accessed via various protocols.

\section{JADE documentation}
Technical noteas are available in hard copies and scanned version in MPP (and DESY library?).
Papers are available on InSpire from CERN library or KEK.
The data preservation documentation is available on wwwjade.mpp.mpg.de.

\section{Conclusions}
The case of JADE data preservation was studied.

\section{TODO}
\begin{itemize}
\item fix this document, 5h
\item Do testst on different platforms with MC, 10h
\item Fix the webpage, 5h
\item Provide VMS., 10h
\end{itemize}



