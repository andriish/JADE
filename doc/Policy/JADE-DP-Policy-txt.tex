\section{Introduction to JADE experiment}
The description of JADE experiment can be found elsewhere.\cite{Bartel:1986ua}
Physics cases for the available JADE data can be discussed elsewhere\cite{Bartel:1986ua}.
The case of JADE data preservation can be found interesting as a mode of data preservation that can be used for other experiments.
The main part of the preservation is the  software, data and documentation.
\section{JADE software}
JADE software description can be found in the JADE computing notes and elsewhere.
Briefly it consists of approximately 50.000 lines of IBM Fortran77 and similar codes 
that were compiled on AIX4.3 systems with IBM Fortran compiler and corresponding runtime.
The dependencies include CERNLIB.


\section{JADE data}
The JADE data is stored in MPCDF.
It can be accessed via various protocols.

The real data is stored in fpack format, which can be converted to the bos format with utility.

\section{JADE documentation}
Technical notes are available in hard copies and scanned version in MPP (and DESY library?).
Papers are available on InSpire from CERN library or KEK.
The data preservation documentation is available on wwwjade.mpp.mpg.de.

\section{Physics cases}
There are limitations on the physics analysis that can be performed with the preserved JADE data.

\section{Authorship}
Any
publication that results from data analysis by non-members of the collaboration will require a
suitable acknowledgement and disclaimer to be included: acknowledgement that the data was
collected by JADE, and disclaimer that no responsibility for the results is taken by the
collaboration. A suitable disclaimer is:

{\it This paper is based on data obtained by the JADE experiment, but is analyzed
independently, and has not been reviewed by the JADE collaboration. }

The analysis can be signed by a collaboration as a whole after a successful approval procedure.
Regardles of that, in case any  member of the JADE collaboration participates 
in such analysis it should be explicitely stated 
if an approval procedure took place and what was the outcome.

The approval procedure should consist of an invitation to all JADE collaboration members 
to participate in the editorial process. The period between the call and  submission of results to a peer-reviewed journal
in any cas should not be shorter than three months.



\section{Conclusions}
The case of JADE data preservation was studied.



