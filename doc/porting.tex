
\documentclass[a4paper,10pt]{article}
\usepackage[left= 3cm, right=3cm]{geometry}
\usepackage[utf8]{inputenc}
\usepackage{placeins}
\usepackage[titletoc,title]{appendix}
\usepackage{authblk}
\usepackage{lineno}
\usepackage{hyperref}
\usepackage[centertags,intlimits]{amsmath}
\usepackage{amssymb,amscd}
\usepackage{calc}
%\usepackage{showframe}
\usepackage{supertabular}
\usepackage{graphicx}
\usepackage{multicol}
\usepackage{multirow}
\usepackage{ifthen}
\usepackage{titlesec}
\usepackage{longtable}
\usepackage{ifthen}
\usepackage{import}
\usepackage{xspace}
\usepackage{authblk}
\usepackage{ifthen}
\usepackage{tikz}
\usepackage{setspace}
\usepackage{multirow}
\usepackage{pgfplots}
\pgfplotsset{compat=1.10} 
%\usepackage[export]{adjustbox}

\renewcommand{\floatpagefraction}{1.99}
%opening

%%%%%%%%%%%%%%%%%%%%%%%%%%%%%%%%%%%%%%%%%%%%%%%%%%%%%%
\newcommand{\eVdist}{\kern-0.06667em}
\newcommand{\GeV}{{\,\text{Ge}\eVdist\text{V\/}}}
\newcommand{\TeV}{{\,\text{Te}\eVdist\text{V\/}}}
\newcommand{\Mev}{{\,\text{Me}\eVdist\text{V\/}}}
\newcommand{\cm}{{\,\text{cm}}}
\newcommand{\epem}{$e^+e^-\,$}
\newcommand{\ep}{$e^{\pm}p\,$}

\newcommand{\PYTHIA}{P}
\newcommand{\TAUOLA}{T}
\newcommand{\KKTF}{K}
\newcommand{\ARIADNE}{A}
\newcommand{\HERWIG}{H}
\newcommand{\GRCFF}{G}
\newcommand{\KORALW}{KW}
\newcommand{\JETSET}{J}

\newcommand{\durham}{Durham\xspace}
\newcommand{\eekt}{Durham\xspace}
\newcommand{\antikt}{anti-$k_T$\xspace}
\newcommand{\eeantikt}{\antikt\xspace}
\newcommand{\meantikt}{\antikt\xspace}
\newcommand{\siscone}{SiScone\xspace}
\newcommand{\miscone}{SiScone\xspace}
\newcommand{\pxcone}{PxCone\xspace}
\newcommand{\jade}{Jade\xspace}
\newcommand{\ca}{Cambridge-Aachen\xspace}
\newcommand{\eecambridge}{\ca}
\newcommand{\durhamcut}{$y_{cut}$}
\newcommand{\eeantiktcut}{$E_{cut}$,\GeV}
\newcommand{\sisconecut}{$E_{cut}$,\GeV}
\newcommand{\pxconecut}{$E_{cut}$,\GeV}
\newcommand{\jadecut}{$y_{cut}$}
\newcommand{\cacut}{$y_{cut}$}
\newcommand{\eecambridgecut}{$sd_{cut}$}


\newcommand{\herwig}{Herwig\xspace}
\newcommand{\pythiasix}{Pythia6\xspace}
\newcommand{\pythiaeight}{Pythia8\xspace}
%%%%%%%%%%%%%%%%%%%%%%%%%%%%%%%%%%%%%%%%%%%%%%%%%%%%%%

\begin{document}
\subsection{Porting to Linux}

So, we have to consider the following systems:
\begin{itemize}
\item AIX? on RS6000  -- not available, the system where the job was done in 2003
\item {
AIX4.3 on RS6000, one machine in MPP. 
\begin{itemize}
\item Big ending, 32? bit
\item g77
\item IBM xlf
\item CERNLIB ?
\item unknown performance 
\end{itemize}
}
\item {
CentOS7 on AMD/Intels x8664, many  real machines, virtual machines  x8664 Virtual Box, x8664 qemu
\begin{itemize}
\item Litle ending, 32 and 64 bit
\item gfortran
\item CERNLIB2006 32 bit/gfortran
\item various peeformance, full compilation less than 3 minutes
\item purpose the target 
\end{itemize}
}

\item {
CentOS7 ppc64, one  virtual machine x8664 qemu
\begin{itemize}
\item Big ending, 32 and 64 bit
\item gfortran
\item IBM xlf (free trial 60 days), available on request/registration
\item xbae  and  CERNLIB2006 32 bit/gfortran should be compiled, 32 bits Xlibs are present in repos
\item Slow, full compilation less takes 15 minutes
\item purpose use for comparison of xlf and gfortran in native BE  environment 
\end{itemize}
}

\item {
CentOS7 ppc64le, one  virtual machine x8664 qemu
\begin{itemize}
\item Litle ending, 32 and 64 bit
\item gfortran
\item IBM xlf  (free trial 60 days), free download
\item CERNLIB2006 32 bit/gfortran??? Not clear how to compile and how many libraries
\item Slow, full compilation less takes 15 minutes
\item purpose: test endieness issues for the LE x8664 porting

\end{itemize}
}
Tested:
x8664+gfortran +CERNLIB,32bit:  
ppc64+gfortran +CERNLIB,32bit:  
ppc64le+gfortran +CERNLIB,32bit:  
ppc64+xlf +CERNLIB,32bit
ppc64le+xlf +CERNLIB,32bit

Because of cernlib issues everything was tested in 32 bit mode.
For all cases hepmc3 works well.
Maybe that would worth to copy the needed cernlib routines to have them separately?
Will minicernlib work?

The best config would be 
x8664+gfortran, no cernlib, 64bit

The best strategy is to split steps of migration
\item {
\begin{itemize}
\item move from  xlf to gfortran
\item move from  big ending to litle ending
\item move from  pcc to x8664
\item move from 32 bit to 64 bit
\item move from cernlib to nocernlib
\end{itemize}
}
However, to reduce the amount of efforts, some steps will be combined.
E.g. it is not visible to get ppc64le+gfortran +CERNLIB,32bit, as it will require to compile a lot of ppc packages
Most likely it will be 
ppc64+xlf+CERNLIB32
ppc64+gfortran+CERNLIB32
x8664+gfortran+CERNLIB32
It is likely nocernlib will not happen.
However, there are very few places where cenlib is used for computing.
The hardest part is graphics.
\end{itemize}

DEPS:
to compile xbae, install ALL *devel* packages and run
%LDFLAGS=' -L/usr/lib   -Wl,-z -Wl,relro' CC='gcc -m32 ' CXX='g++ -m32 '   rpmbuild --target=ppc  --rebuild xbae-4.60.4-12.el6.src.rpm -D'_lib lib' -D'libdir /usr/lib'
No idea if %-L/usr/lib 
is really needed with all packages, but it runs!

In addition, a brute force trick is to replace /usr/bin/gfortran with a wrapper with explicit -m32 -L/usr/lib flags.
Same trick for /usr/bin/rm:   add -f option.
Actually needed only because batch builds would require input y when deleting files

The slowness issue can be solved, maybe, with qemu-2.7 with multitreading emulation.
Maybe the earlier versions have it as well. No idea.


In the same way cernlib is compiled

To compile cernlib nohup is usefull. The VM is slow and can drop connection







\end{document}
